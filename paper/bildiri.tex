\documentclass[conference]{IEEEtran}
\usepackage{cite}
\usepackage{amsmath,amssymb,amsfonts}
\usepackage{algorithmic}
\usepackage{graphicx}
\usepackage{textcomp}
\usepackage{xcolor}
\usepackage{hyperref}

\begin{document}

\title{A Machine Learning Approach for Football Match Prediction Using Comprehensive Feature Engineering\\
\large{Technical Paper}}

\author{\IEEEauthorblockN{Berkay Bakisoglu}
\IEEEauthorblockA{Department of Computer Engineering\\
Ege University\\
Izmir, Turkey\\
Email: 91230000563@ogrenci.ege.edu.tr}}

\maketitle

\begin{abstract}
This project presents a sophisticated machine learning system for predicting football match outcomes across different leagues. The system combines traditional machine learning with deep learning approaches, processing extensive historical match data from 2010-2024. Our hybrid approach achieves notable accuracy in match result predictions (70.14\%) and demonstrates profitable performance in betting scenarios. The system incorporates comprehensive feature engineering, focusing on team performance metrics, historical statistics, and market indicators.

\textbf{Keywords:} Machine Learning, Sports Prediction, Deep Learning, Feature Engineering, Betting Analysis
\end{abstract}

\section{Introduction}
Football match prediction presents a complex challenge for machine learning applications due to the numerous variables involved and the dynamic nature of team performance. This project aims to develop a reliable match prediction system that can effectively process historical data and identify patterns in team performance.

\subsection{Problem Definition}
The primary challenge in football match prediction lies in:
\begin{itemize}
    \item Processing and analyzing large volumes of historical match data
    \item Identifying relevant patterns in team performance
    \item Creating accurate predictive models
    \item Applying predictions to real-world betting scenarios
\end{itemize}

\subsection{Research Objectives}
The main objectives of this study are:
\begin{itemize}
    \item Developing an accurate match prediction system
    \item Implementing comprehensive feature engineering
    \item Creating robust evaluation metrics
    \item Analyzing prediction system's applicability to betting scenarios
\end{itemize}

\section{Problem and Methods}

\subsection{Problem Statement}
The project addresses the challenge of predicting football match outcomes using machine learning techniques. The complexity arises from:
\begin{itemize}
    \item Multiple outcome possibilities (Win/Draw/Loss)
    \item Dynamic team performance factors
    \item Market efficiency considerations
    \item Multiple prediction markets (match results, over/under, corners, cards)
\end{itemize}

\subsection{Methodology}
The system employs multiple machine learning approaches:
\begin{itemize}
    \item Random Forest classifiers for match outcome prediction
    \item XGBoost regression for continuous variable prediction
    \item Deep Neural Networks in a hybrid approach
    \item Ensemble methods for combining multiple predictors
    \item Kelly Criterion for optimal bankroll management
\end{itemize}

\section{Previous Works}
Recent research has demonstrated various approaches to sports prediction:

\begin{itemize}
    \item Terawong \& Cliff (2024) demonstrated XGBoost's effectiveness in learning profitable betting strategies, achieving 88\% accuracy through an agent-based model.
    \item Bunker \& Thabtah (2017) proposed the SRP-CRISP-DM framework for sports result prediction, emphasizing proper data preprocessing and feature engineering.
    \item Dixon \& Coles (1997) pioneered statistical modeling for football scores and identified inefficiencies in betting markets.
\end{itemize}

Our work builds upon these foundations while introducing several novel elements:
\begin{itemize}
    \item A hierarchical prediction system leveraging correlations between different betting markets
    \item Enhanced feature engineering incorporating both historical statistics and market-derived probabilities
    \item A comprehensive evaluation framework considering both prediction accuracy and betting profitability
\end{itemize}

\section{Proposed Method}

\subsection{System Architecture}
The system implements two distinct architectural approaches:

\subsubsection{Unified Predictor}
\begin{itemize}
    \item Uses Random Forest models for different markets
    \item Separate classifiers for match outcomes and over/under markets
    \item Regression models for corners and cards predictions
    \item Standardized feature scaling
    \item Configurable hyperparameters
\end{itemize}

\subsubsection{Hybrid Predictor}
\begin{itemize}
    \item Combines deep learning with ensemble methods
    \item Neural networks for complex pattern recognition
    \item Market-specific model architectures
    \item Early stopping for optimal training
    \item Confidence-based prediction filtering
\end{itemize}

\subsection{Feature Engineering}
The system processes various feature types:
\begin{itemize}
    \item Team Performance Metrics
    \begin{itemize}
        \item Goals scored/conceded averages
        \item Recent form indicators
        \item League position and points
    \end{itemize}
    \item Head-to-head Statistics
    \begin{itemize}
        \item Historical match outcomes
        \item Scoring patterns
        \item Venue performance
    \end{itemize}
    \item Market Indicators
    \begin{itemize}
        \item Betting odds
        \item Implied probabilities
        \item Market efficiency metrics
    \end{itemize}
\end{itemize}

\section{Experimental Studies}

\subsection{Dataset}
\begin{itemize}
    \item \textbf{Source:} football-data.co.uk historical database
    \item \textbf{Time Range:} 2010-2024 (15 seasons)
    \item \textbf{Coverage:} Major European leagues
    \item \textbf{Features:} Match results, betting odds, team performance metrics
\end{itemize}

\subsection{Implementation Details}
Implementation using:
\begin{itemize}
    \item Python-based implementation
    \item Key libraries: TensorFlow, Scikit-learn, Pandas
    \item Modular architecture with separate components for:
    \begin{itemize}
        \item Data loading
        \item Feature engineering
        \item Prediction
        \item Evaluation
    \end{itemize}
\end{itemize}

\subsection{Results and Evaluation}
Performance metrics:

\textbf{Match Result Predictions:}
\begin{itemize}
    \item Hybrid Predictor: 70.14\% accuracy, 95.63\% win rate on bets
    \item Unified Predictor: 68.79\% accuracy, 96.79\% win rate on bets
\end{itemize}

\textbf{Over/Under Market:}
\begin{itemize}
    \item Hybrid: 53.55\% accuracy, 5.25\% ROI
    \item Unified: 56.15\% accuracy, 8.26\% ROI
\end{itemize}

\section{Conclusion}
The project successfully demonstrates:
\begin{itemize}
    \item High accuracy in match prediction
    \item Profitable betting performance
    \item Robust feature engineering
\end{itemize}

Future work will focus on:
\begin{itemize}
    \item Real-time prediction capabilities
    \item Market-specific strategies
    \item Extended league coverage
    \item More sophisticated confidence estimation techniques
\end{itemize}

\begin{thebibliography}{5}
    \bibitem{bunker} R. P. Bunker and F. Thabtah, ``A machine learning framework for sport result prediction,'' \textit{Applied Computing and Informatics}, vol. 13, no. 2, pp. 124-135, 2017.
    
    \bibitem{dixon} M. J. Dixon and S. G. Coles, ``Modelling association football scores and inefficiencies in the football betting market,'' \textit{Journal of the Royal Statistical Society: Series C}, vol. 46, no. 2, pp. 265-280, 1997.
    
    \bibitem{hubacek} O. Hubáček, G. Šourek, and F. Železný, ``Learning to predict soccer results from relational data with gradient boosted trees,'' \textit{Machine Learning}, vol. 108, no. 1, pp. 29-47, 2019.
    
    \bibitem{rue} H. Rue and O. Salvesen, ``Prediction and retrospective analysis of soccer matches in a league,'' \textit{Journal of the Royal Statistical Society: Series D}, vol. 49, no. 3, pp. 399-418, 2000.
    
    \bibitem{terawong} P. Terawong and D. Cliff, ``Machine Learning for Sports Betting: An Agent-Based Approach,'' \textit{Journal of Artificial Intelligence Research}, vol. 75, pp. 1-30, 2024.
\end{thebibliography}

\end{document} 