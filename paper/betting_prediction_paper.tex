\documentclass[conference]{IEEEtran}
\usepackage{cite}
\usepackage{amsmath,amssymb,amsfonts}
\usepackage{algorithmic}
\usepackage{graphicx}
\usepackage{textcomp}
\usepackage{xcolor}
\usepackage{hyperref}
\usepackage{tikz}
\usetikzlibrary{shapes,arrows,positioning,fit}

\begin{document}

\title{A Machine Learning Approach for Football Match Prediction Using Comprehensive Feature Engineering\\
\large{Preliminary Report}}

\author{\IEEEauthorblockN{Berkay Bakisoglu}
\IEEEauthorblockA{Department of Computer Engineering\\
Ege University\\
Izmir, Turkey\\
Email: 91230000563@ogrenci.ege.edu.tr}}

\maketitle

\begin{abstract}
In this paper, we explore the development of a machine learning-based match prediction system for football. Our primary goal is to create an accurate prediction system that can effectively forecast match outcomes across different leagues. We propose a comprehensive approach that combines feature engineering with various machine learning models, processing extensive historical match data (2010-2024) with particular attention to team performance patterns and historical statistics. While this is a preliminary report, our initial analysis suggests promising directions not only for accurate match prediction but also for potential applications in betting markets. We discuss our methodology, current progress, and planned experiments for validating our approach.
\end{abstract}

\section{Introduction}
Football match prediction presents a complex challenge for machine learning applications. Despite the abundance of historical match data and statistics, accurately predicting match outcomes remains difficult due to the numerous variables involved and the dynamic nature of team performance. Our research began with the goal of developing a reliable match prediction system that can effectively process historical data and identify patterns in team performance. Beyond the primary goal of accurate prediction, we also aim to explore whether such a system could be effectively applied in betting scenarios.

\subsection{Problem Statement}
Through our initial research, we identified several key challenges:
\begin{itemize}
\item Extracting meaningful patterns from complex match statistics and historical data
\item Developing models that can adapt to changing team performance and form
\item Creating effective evaluation methods that consider both prediction accuracy and consistency
\item Exploring the practical applications of accurate predictions in betting contexts
\end{itemize}

\subsection{Research Objectives}
Our research has two main goals:

\subsubsection{Primary Objectives}
\begin{itemize}
\item Development of an accurate match prediction system for multiple leagues
\item Implementation of comprehensive feature engineering focusing on team performance metrics
\item Creation of robust evaluation metrics for prediction accuracy
\item Analysis of different machine learning approaches for match prediction
\end{itemize}

\subsubsection{Secondary Objectives}
\begin{itemize}
\item Evaluation of prediction system's applicability to betting scenarios
\item Analysis of prediction confidence in relation to betting decisions
\item Investigation of system performance in different betting markets
\end{itemize}

\section{Previous Works}
Sports betting prediction using machine learning has gained significant attention in recent years. Terawong \& Cliff (2024) demonstrated the effectiveness of XGBoost in learning profitable betting strategies through an agent-based model of a sports betting exchange. Their work showed that machine learning models could learn strategies that outperform traditional betting approaches, achieving an overall accuracy of 88\%.

Bunker \& Thabtah (2017) proposed a structured framework for sports result prediction (SRP-CRISP-DM) that emphasizes the importance of proper data preprocessing and feature engineering. Their framework distinguishes between match-related and external features, and advocates for preserving temporal order in model evaluation.

These works highlight two key aspects in sports betting prediction: the importance of sophisticated machine learning approaches and the need for proper data handling and evaluation methodologies. Our work builds upon these foundations while introducing several novel elements:

\begin{itemize}
    \item A hierarchical prediction system that leverages correlations between different betting markets
    \item Enhanced feature engineering incorporating both historical statistics and market-derived probabilities
    \item A comprehensive evaluation framework that considers both prediction accuracy and betting profitability
\end{itemize}

\section{Methodology}
Our approach introduces a novel hierarchical prediction system for football betting markets. The system consists of four main components:

\subsection{Data Processing}
The data processing pipeline handles multiple seasons of football match data from various European leagues. Our dataset contains comprehensive match statistics and betting odds, including:

\begin{itemize}
    \item \textbf{Match Statistics}: Full-time (FT) and half-time (HT) scores, shots on target (HST/AST), fouls (HF/AF), corners (HC/AC), and cards (HY/AY/HR/AR)
    \item \textbf{Betting Odds}: Pre-match odds from multiple bookmakers (Bet365, BetWin, Betfair, etc.) for:
        \begin{itemize}
            \item Match results (Home/Draw/Away)
            \item Over/Under 2.5 goals
            \item Asian Handicap markets
            \item Corner markets
        \end{itemize}
\end{itemize}

\subsection{Feature Engineering}
Our feature engineering pipeline creates sophisticated predictive features from the raw data:

\begin{itemize}
    \item \textbf{Team Performance Features}:
        \begin{itemize}
            \item Recent form with exponential decay weights
            \item Rolling averages for goals scored/conceded
            \item Team-specific metrics (clean sheets, scoring patterns)
            \item Days since last game for recovery analysis
        \end{itemize}
    
    \item \textbf{Market-Derived Features}:
        \begin{itemize}
            \item Implied probabilities from betting odds
            \item Market overround calculations
            \item Value betting indicators
            \item Favorite/underdog identification
        \end{itemize}
    
    \item \textbf{League Position Features}:
        \begin{itemize}
            \item Dynamic league standings
            \item Points and position differences
            \item Goal difference rankings
            \item Form-based league performance
        \end{itemize}
    
    \item \textbf{Match-Specific Features}:
        \begin{itemize}
            \item Head-to-head statistics
            \item Home/Away performance metrics
            \item Referee influence indicators
            \item Derby match identification
        \end{itemize}
    
    \item \textbf{Advanced Statistical Features}:
        \begin{itemize}
            \item Standard deviation of performance metrics
            \item Weighted historical encounters
            \item Seasonal progression indicators
            \item League-specific normalization
        \end{itemize}
\end{itemize}

The feature engineering process includes careful handling of temporal aspects to prevent data leakage, with all features calculated using only historical data available before each match. Feature importance analysis guides the selection of most relevant predictors for each market.

\subsection{Hierarchical Prediction System}
The core of our system is a novel hierarchical approach that combines multiple prediction models:

\begin{itemize}
    \item \textbf{Base Predictors}: Specialized models for corners and cards using LightGBM and Random Forest
    \item \textbf{Enhanced Match Predictor}: A final classifier that incorporates predictions from base models
    \item \textbf{Confidence Estimation}: Uncertainty quantification using model-specific techniques
\end{itemize}

\subsection{Evaluation Framework}
Our evaluation framework provides comprehensive assessment through:

\begin{itemize}
    \item \textbf{Time-Series Validation}: Proper handling of temporal dependencies
    \item \textbf{Market-Specific Metrics}: Tailored evaluation metrics for each prediction target
    \item \textbf{League-Specific Analysis}: Performance breakdown by league and season
    \item \textbf{Visualization Tools}: Advanced plotting capabilities for result analysis
\end{itemize}

\section{Experimental Studies}
\subsection{Dataset and Implementation Details}
\subsubsection{Dataset Characteristics}
Our dataset comprises historical football match data from major European leagues:
\begin{itemize}
\item \textbf{Data Source}: Historical match data and betting odds from 2010 to 2024, football-data.co.uk
\item \textbf{Dataset Size}: 15 seasons of data across multiple leagues
\item \textbf{Features}: Over 30 features per match including:
    \begin{itemize}
    \item Match statistics (goals, corners, cards)
    \item Team performance metrics
    \item Historical head-to-head records
    \item Betting odds from Bet365
    \end{itemize}
\item \textbf{File Format}: Mix of .xls and .xlsx files, organized by season and league
\end{itemize}

\subsubsection{Implementation Environment}
The system was implemented using the following technologies:
\begin{itemize}
\item \textbf{Programming Language}: Python 3.10+
\item \textbf{Key Libraries}:
    \begin{itemize}
    \item scikit-learn (for Random Forest models)
    \item LightGBM (for gradient boosting)
    \item pandas (for data manipulation)
    \item numpy (for numerical operations)
    \item matplotlib and seaborn (for visualization)
    \end{itemize}
\item \textbf{Development Environment}: Pycharm
\end{itemize}

\subsection{Training Methodology}
\subsubsection{Data Preprocessing}
Our preprocessing pipeline includes:
\begin{itemize}
\item Temporal alignment of match data
\item Feature scaling using StandardScaler
\item Handling of missing values through predefined rules
\item Validation of data completeness and consistency
\end{itemize}

\subsubsection{Cross-Validation Strategy}
We implemented a time-series based cross-validation approach:
\begin{itemize}
\item \textbf{Training Mode}:
    \begin{itemize}
    \item Minimum 3 seasons required for validation
    \item Sliding window approach for season selection
    \item Sequential split to maintain temporal order
    \end{itemize}
\item \textbf{Test Mode}:
    \begin{itemize}
    \item 80\% training, 20\% testing split
    \item Configurable test size parameter
    \item Rapid prototyping capabilities
    \end{itemize}
\end{itemize}

\subsection{Model Parameters}
\subsubsection{Random Forest Configuration}
For classification tasks (match results):
\begin{itemize}
\item n\_estimators: 200
\item max\_depth: 15
\item min\_samples\_split: 10
\item min\_samples\_leaf: 5
\item class\_weight: 'balanced'
\end{itemize}

\subsubsection{LightGBM Configuration}
For regression tasks (corner-card prediction):
\begin{itemize}
\item n\_estimators: 500
\item learning\_rate: 0.01
\item num\_leaves: 31
\item max\_depth: 8
\item min\_child\_samples: 20
\end{itemize}

\subsection{Performance Metrics}
We evaluate our models using multiple metrics:
\begin{itemize}
\item \textbf{Classification Tasks}:
    \begin{itemize}
    \item Accuracy
    \item Precision
    \item Recall
    \item F1-score
    \end{itemize}
\item \textbf{Regression Tasks}:
    \begin{itemize}
    \item Mean Squared Error (MSE)
    \item Root Mean Squared Error (RMSE)
    \item Mean Absolute Error (MAE)
    \item R-squared (R2)
    \end{itemize}
\end{itemize}

\subsection{Analysis Tools}
Our evaluation framework includes:
\begin{itemize}
\item \textbf{Seasonal Progression Analysis}:
    \begin{itemize}
    \item Tracking prediction accuracy over time
    \item Analyzing model adaptation to season changes
    \end{itemize}
\item \textbf{League-Specific Analysis}:
    \begin{itemize}
    \item Performance comparison across leagues
    \item League-specific feature importance
    \end{itemize}
\item \textbf{Feature Importance Analysis}:
    \begin{itemize}
    \item Ranking of most influential features
    \item Market-specific feature analysis
    \end{itemize}
\end{itemize}

\section{Experimental Results}
\subsection{Preliminary Setup}
Our experimental framework addresses both prediction accuracy and betting applications:

\subsubsection{Planned Experiments}
We aim to answer several key questions:
\begin{itemize}
\item How do different models perform in predicting match outcomes?
\item Which features are most important for accurate prediction?
\item How does prediction accuracy vary across different leagues?
\item What is the impact of historical data window size on prediction accuracy?
\item How well do accurate predictions translate to betting success?
\item Which types of predictions offer the best betting opportunities?
\end{itemize}

\subsubsection{Initial Data Analysis}
Our preliminary investigation has focused on understanding our data:
\begin{itemize}
\item Analyzing match outcome distributions across leagues
\item Understanding team performance patterns
\item Examining Bet365 odds patterns and distributions
\item Evaluating feature correlations with match outcomes
\end{itemize}

\subsection{Next Steps}
Our immediate plans include:
\begin{itemize}
\item Testing different model architectures
\item Analyzing feature importance across leagues
\item Validating prediction accuracy with historical data
\item Comparing our predictions with Bet365 implied probabilities
\item Evaluating prediction confidence thresholds
\item Testing performance using historical Bet365 odds
\end{itemize}

\section{Conclusion}
This preliminary report represents our first steps toward developing an effective machine learning approach to football match prediction, with potential applications in betting markets. While we're still early in our research, our initial work has revealed both promising directions and significant challenges.

\subsection{Current Progress}
We've established some fundamental building blocks:
\begin{itemize}
\item A robust approach to processing historical match data
\item A framework for extracting meaningful performance features
\item Initial prediction model architectures
\item A comprehensive evaluation methodology
\end{itemize}

\subsection{Future Work}
Our next phase will focus on:
\begin{itemize}
\item Implementing and testing various prediction models
\item Conducting extensive validation experiments
\item Expanding our literature review
\item Analyzing prediction performance across different leagues
\end{itemize}

\end{document} 