\documentclass[conference]{IEEEtran}
\usepackage{cite}
\usepackage{amsmath,amssymb,amsfonts}
\usepackage{algorithmic}
\usepackage{graphicx}
\usepackage{textcomp}
\usepackage{xcolor}
\usepackage{hyperref}
\usepackage{tikz}
\usetikzlibrary{shapes,arrows,positioning,fit}

\begin{document}

\title{A Machine Learning Approach for Football Match Prediction Using Comprehensive Feature Engineering\\
\large{Preliminary Report}}

\author{\IEEEauthorblockN{Berkay Bakisoglu}
\IEEEauthorblockA{Department of Computer Engineering\\
Ege University\\
Izmir, Turkey\\
Email: 91230000563@ogrenci.ege.edu.tr}}

\maketitle

\begin{abstract}
In this paper, we explore the development of a machine learning-based match prediction system for football. Our primary goal is to create an accurate prediction system that can effectively forecast match outcomes across different leagues. We propose a comprehensive approach that combines feature engineering with various machine learning models, processing extensive historical match data (2010-2024) with particular attention to team performance patterns and historical statistics. While this is a preliminary report, our initial analysis suggests promising directions not only for accurate match prediction but also for potential applications in betting markets. We discuss our methodology, current progress, and planned experiments for validating our approach.
\end{abstract}

\section{Introduction}
Football match prediction presents a complex challenge for machine learning applications. Despite the abundance of historical match data and statistics, accurately predicting match outcomes remains difficult due to the numerous variables involved and the dynamic nature of team performance. Our research began with the goal of developing a reliable match prediction system that can effectively process historical data and identify patterns in team performance. Beyond the primary goal of accurate prediction, we also aim to explore whether such a system could be effectively applied in betting scenarios.

\subsection{Problem Statement}
Through our initial research, we identified several key challenges:
\begin{itemize}
\item Extracting meaningful patterns from complex match statistics and historical data
\item Developing models that can adapt to changing team performance and form
\item Creating effective evaluation methods that consider both prediction accuracy and consistency
\item Exploring the practical applications of accurate predictions in betting contexts
\end{itemize}

\subsection{Research Objectives}
Our research has two main goals:

\subsubsection{Primary Objectives}
\begin{itemize}
\item Development of an accurate match prediction system for multiple leagues
\item Implementation of comprehensive feature engineering focusing on team performance metrics
\item Creation of robust evaluation metrics for prediction accuracy
\item Analysis of different machine learning approaches for match prediction
\end{itemize}

\subsubsection{Secondary Objectives}
\begin{itemize}
\item Evaluation of prediction system's applicability to betting scenarios
\item Analysis of prediction confidence in relation to betting decisions
\item Investigation of system performance in different betting markets
\end{itemize}

\section{Previous Works}
Sports betting prediction using machine learning has gained significant attention in recent years. Terawong \& Cliff (2024) demonstrated the effectiveness of XGBoost in learning profitable betting strategies through an agent-based model of a sports betting exchange. Their work showed that machine learning models could learn strategies that outperform traditional betting approaches, achieving an overall accuracy of 88\%.

Bunker \& Thabtah (2017) proposed a structured framework for sports result prediction (SRP-CRISP-DM) that emphasizes the importance of proper data preprocessing and feature engineering. Their framework distinguishes between match-related and external features, and advocates for preserving temporal order in model evaluation.

These works highlight two key aspects in sports betting prediction: the importance of sophisticated machine learning approaches and the need for proper data handling and evaluation methodologies. Our work builds upon these foundations while introducing several novel elements:

\begin{itemize}
    \item A hierarchical prediction system that leverages correlations between different betting markets
    \item Enhanced feature engineering incorporating both historical statistics and market-derived probabilities
    \item A comprehensive evaluation framework that considers both prediction accuracy and betting profitability
\end{itemize}

\section{Methodology}
This section presents our approach to football match prediction, which combines standardized feature engineering with three distinct model architectures. The methodology is structured to enable fair comparison between different prediction approaches while maintaining architectural innovation.

\subsection{Feature Engineering}
The system employs a standardized set of features across all predictors, ensuring fair model comparison. Features are organized into:

\subsubsection{Base Features}
Team performance metrics including goals scored/conceded, clean sheets, and win rates. Form features capture recent performance through exponentially weighted averages. League position features track team standings and points.

\subsubsection{Market Features}
Market-derived features include implied probabilities and value indicators for match outcomes (Home/Draw/Away) and total goals (Over/Under 2.5). Market confidence metrics and overround calculations assess betting efficiency.

\subsection{Prediction Systems}
Three distinct architectural approaches share the standardized feature set:

\subsubsection{Unified Predictor}
Combines Random Forest and LightGBM models for independent predictions across markets. Handles both match outcomes and over/under predictions through separate classifiers.

\subsubsection{Hierarchical Predictor}
Sequential LightGBM models predict auxiliary markets first (cards, corners), enhancing match outcome predictions. Maintains separate models for over/under predictions.

\subsubsection{Weighted XGBoost Predictor}
Employs market-odds weighted XGBoost models. Handles match outcomes and over/under markets with specialized sample weights based on implied probabilities.

\subsection{Evaluation Framework}
Performance assessment includes:
\begin{itemize}
    \item Classification metrics (accuracy, precision, recall) for match outcomes and over/under predictions
    \item Return on Investment (ROI) calculations for each market
    \item Value bet identification based on predicted probabilities versus market odds
    \item Market-specific analysis comparing performance across different betting types
\end{itemize}

\subsection{Model Comparison Framework}
To ensure scientific rigor in comparing these architectures:
\begin{itemize}
    \item All models use identical feature sets
    \item Time-series based validation preserves temporal order
    \item Performance metrics include accuracy, ROI, and market-specific measures
    \item Analysis of model strengths across different leagues and seasons
\end{itemize}

This methodology enables us to evaluate the effectiveness of different architectural approaches while maintaining consistency in feature engineering and evaluation metrics.

\section{Experimental Studies}
\subsection{Dataset and Implementation Details}
\subsubsection{Dataset Characteristics}
Our dataset comprises historical football match data from major European leagues:
\begin{itemize}
\item \textbf{Data Source}: Historical match data and betting odds from 2010 to 2024, football-data.co.uk
\item \textbf{Dataset Size}: 15 seasons of data across multiple leagues
\item \textbf{Features}: Over 30 features per match including:
    \begin{itemize}
    \item Match statistics (goals, corners, cards)
    \item Team performance metrics
    \item Historical head-to-head records
    \item Betting odds from Bet365
    \end{itemize}
\item \textbf{File Format}: Mix of .xls and .xlsx files, organized by season and league
\end{itemize}

\subsubsection{Implementation Environment}
The system was implemented using the following technologies:
\begin{itemize}
\item \textbf{Programming Language}: Python 3.10+
\item \textbf{Key Libraries}:
    \begin{itemize}
    \item scikit-learn (for Random Forest models)
    \item LightGBM (for gradient boosting)
    \item pandas (for data manipulation)
    \item numpy (for numerical operations)
    \item matplotlib and seaborn (for visualization)
    \end{itemize}
\item \textbf{Development Environment}: Pycharm
\end{itemize}

\subsection{Training Methodology}
\subsubsection{Data Preprocessing}
Our preprocessing pipeline includes:
\begin{itemize}
\item Temporal alignment of match data
\item Feature scaling using StandardScaler
\item Handling of missing values through predefined rules
\item Validation of data completeness and consistency
\end{itemize}

\subsubsection{Cross-Validation Strategy}
We implemented a time-series based cross-validation approach:
\begin{itemize}
\item \textbf{Training Mode}:
    \begin{itemize}
    \item Minimum 3 seasons required for validation
    \item Sliding window approach for season selection
    \item Sequential split to maintain temporal order
    \end{itemize}
\item \textbf{Test Mode}:
    \begin{itemize}
    \item 80\% training, 20\% testing split
    \item Configurable test size parameter
    \item Rapid prototyping capabilities
    \end{itemize}
\end{itemize}

\subsection{Performance Metrics}
We evaluate our models using multiple metrics:
\begin{itemize}
\item \textbf{Prediction Accuracy Metrics}:
    \begin{itemize}
    \item Accuracy: Overall prediction accuracy for match outcomes
    \item Precision: Accuracy of positive predictions
    \item Recall: Ability to detect positive cases
    \item F1-score: Harmonic mean of precision and recall
    \end{itemize}
\item \textbf{Betting Performance Metrics}:
    \begin{itemize}
    \item Return on Investment (ROI): Percentage return on placed bets
    \item Profit/Loss: Absolute monetary performance
    \item Strike Rate: Percentage of successful bets
    \item Value Betting Analysis: Comparison of predicted vs. market probabilities
    \end{itemize}
\item \textbf{Market-Specific Metrics}:
    \begin{itemize}
    \item Match Results: Classification metrics with ROI per outcome
    \item Corners/Cards: RMSE and MAE for count predictions
    \item Market Efficiency: Analysis of odds-implied vs. predicted probabilities
    \end{itemize}
\end{itemize}

These metrics provide a comprehensive view of both predictive accuracy and betting effectiveness, enabling evaluation of models from both statistical and practical perspectives.

\subsection{Analysis Tools}
Our evaluation framework includes:
\begin{itemize}
\item \textbf{Seasonal Progression Analysis}:
    \begin{itemize}
    \item Tracking prediction accuracy over time
    \item Analyzing model adaptation to season changes
    \end{itemize}
\item \textbf{League-Specific Analysis}:
    \begin{itemize}
    \item Performance comparison across leagues
    \item League-specific feature importance
    \end{itemize}
\item \textbf{Feature Importance Analysis}:
    \begin{itemize}
    \item Ranking of most influential features
    \item Market-specific feature analysis
    \end{itemize}
\end{itemize}

\section{Experimental Results}
\subsection{Preliminary Setup}
Our experimental framework addresses both prediction accuracy and betting applications:

\subsubsection{Planned Experiments}
We aim to answer several key questions:
\begin{itemize}
\item How do different models perform in predicting match outcomes?
\item Which features are most important for accurate prediction?
\item How does prediction accuracy vary across different leagues?
\item What is the impact of historical data window size on prediction accuracy?
\item How well do accurate predictions translate to betting success?
\item Which types of predictions offer the best betting opportunities?
\end{itemize}

\subsubsection{Initial Data Analysis}
Our preliminary investigation has focused on understanding our data:
\begin{itemize}
\item Analyzing match outcome distributions across leagues
\item Understanding team performance patterns
\item Examining Bet365 odds patterns and distributions
\item Evaluating feature correlations with match outcomes
\end{itemize}

\section{Conclusion}
This preliminary report represents our first steps toward developing an effective machine learning approach to football match prediction, with potential applications in betting markets. While we're still early in our research, our initial work has revealed both promising directions and significant challenges.

\subsection{Current Progress}
We've established some fundamental building blocks:
\begin{itemize}
\item A robust approach to processing historical match data
\item A framework for extracting meaningful performance features
\item Initial prediction model architectures
\item A comprehensive evaluation methodology
\end{itemize}

\subsection{Future Work}
Our next phase of research will focus on several key areas:

\subsubsection{Model Development}
\begin{itemize}
\item Testing and comparing different model architectures
\item Implementing ensemble methods for improved prediction accuracy
\item Developing more sophisticated confidence estimation techniques
\end{itemize}

\subsubsection{Validation and Testing}
\begin{itemize}
\item Testing performance using historical betting odds
\end{itemize}

\subsubsection{Betting Applications}
\begin{itemize}
\item Analyzing prediction accuracy in relation to betting profitability
\end{itemize}

\subsubsection{Research Extensions}
\begin{itemize}
\item Expanding our literature review and theoretical foundation
\item Analyzing prediction performance across different leagues and seasons
\item Investigating market-specific prediction strategies
\item Developing real-time prediction capabilities
\end{itemize}

\end{document} 